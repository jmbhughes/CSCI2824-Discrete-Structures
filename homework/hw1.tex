%%%%%%%%%%%%%%%%%%%%%%%%%%%%%%%%%%%%%%%%%%%%%%%%%%%%%%%%%%%%%%%%%%%%%%%%%%%%%%%%%%%%
% Do not alter this block (unless you're familiar with LaTeX
\documentclass{article}
\usepackage[margin=1in]{geometry} 
\usepackage{amsmath,amsthm,amssymb,amsfonts, fancyhdr, color, comment, graphicx, environ}
\usepackage{xcolor}
\usepackage{mdframed}
\usepackage[shortlabels]{enumitem}
\usepackage{indentfirst}
\usepackage{hyperref}
\hypersetup{
    colorlinks=true,
    linkcolor=blue,
    filecolor=magenta,      
    urlcolor=blue,
}


\pagestyle{fancy}


\newenvironment{problem}[2][Problem]
    { \begin{mdframed}[backgroundcolor=gray!20] \textbf{#1 #2} \\}
    {  \end{mdframed}}

% Define solution environment
\newenvironment{solution}
    {\textit{Proof:}}
    {}

\renewcommand{\qed}{\quad\qedsymbol}

% prevent line break in inline mode
\binoppenalty=\maxdimen
\relpenalty=\maxdimen

%%%%%%%%%%%%%%%%%%%%%%%%%%%%%%%%%%%%%%%%%%%%%
%Fill in the appropriate information below
\lhead{Your Name: }
\rhead{CSCI 2824-310} 
\chead{\textbf{Homework 1  Due: 10 June 2019, start of class}}
%%%%%%%%%%%%%%%%%%%%%%%%%%%%%%%%%%%%%%%%%%%%%

\begin{document}
\begin{problem}{Extra Credit}
\textbf{[5pts]} LaTex all your answers! If you don't know and want help, post on Piazza. The template is \href{https://www.overleaf.com/read/gnqfhyjnrysh}{here}.  
\end{problem}

\begin{problem}{1.1}
\textbf{[15 pts]} For each of the following statements, determine whether they are logically equivalent. In each case, construct a truth table and include a sentence justifying your answer. Your sentence should illustrate that you understand the meaning of logical equivalence. Also, show steps using rules of logic to transform the first statement into the second if they are logically equivalent. 
\begin{enumerate}[a)]
    \item $\neg(p \land q)$ and $\neg p \land \neg q$
    \item $p \land (q \lor r)$ and $(p \land q) \lor (p \land r)$
    \item $(p \lor q) \lor (p \land r)$ and $(p \lor q) \land r$
\end{enumerate}
\end{problem}

\begin{solution}
% your solution here
\end{solution}
\newpage  % delete if you want to remove space

%%% Problem 2
\begin{problem}{1.2}
\textbf{[12pts]} Use truth tables to establish the truth of each statement:
\begin{enumerate}[a)]
\item A conditional statement is not logically equivalent to its converse.
\item A conditional statement is not logically equivalent to its inverse.
\item A conditional statement and its contrapositive are logically equivalent to each other.
\item The converse and inverse of a conditional statement are logically equivalent to each other.
\end{enumerate}
\end{problem}

\begin{solution}
% your solution here
\end{solution}
\newpage % delete if you want to remove space

%%% Problem 3
\begin{problem}{1.3}
\textbf{[15pts]} Consider the following satisfiability problem. Fred, Daphne, Shaggy, Velma, and Scooby are going to a movie and they find a set of four seats in the front row for them to sit. Scooby is happy to lounge at their feet. Some of them are bickering and don't want to sit next to each other. Others are happy and do want to sit next to each other. In particular, 
\begin{enumerate}[i.]
    \item Shaggy wants to sit next to Daphne or Velma.
    \item Velma wants to sit next to Daphne, but refuses to sit next to Shaggy or Fred.
    \item If Daphne doesn't sit next to Fred, then Velma wants to sit next to Fred.
    \item Fred wants to sit next to Daphne if and only if Daphne does not sit next to Shaggy.
\end{enumerate}

Let $N(x,y)$ represent the propositional function \textit{``$x$ and $y$ are sitting next to each other,"} where the domain for $x$ and $y$ is the set $\{F, D, S, V\}$ (the individuals). 

\vspace{1em}

Translate each state from English into formal symbolism. Are the group's seating requirements satisfiable? If they are, provide a seating that satisfies the requirements. If not, provide a concise written argument explaining why they are not.

\vspace{1em}

\textbf{Hint}: $N$ is symmetric, i.e. $N(x,y)$ is true if and only if $N(y,x)$ is true. For instance, if Fred and Velma sit next to each other then the propositions $N(F, V) = N(V, F) = \text{True}$. 

\end{problem}

\begin{solution}
% your answer here
\end{solution}
\newpage  % delete if you want to remove space

%%% Problem 4
\begin{problem}{1.4}
\textbf{[15pts]} Sharky, a leader of the underworld, was killed by one of his own band of four henchmen. Detective Sharp interviewed the men and determined that \textbf{all were lying except for one}. Unfortunately, Detective Sharp keeps terrible notes and was killed to cover up the crime. You only have the following statements:
\begin{itemize}
    \item Socko says, ``Lefty killed Sharky."
    \item Fats says, ``Muscles didn't kill Sharky."
    \item Lefty says, ``Muscles was shooting craps with Socko when Sharky was knocked off."
    \item Muscles says, ``Lefty didn't kill Sharky."
\end{itemize}
Who did kill Sharky? Convey your argument in plain English or symbolically. 
\end{problem}

\begin{solution}
\end{solution}
\newpage % delete if you want to remove white space

%%% Problem 5
\begin{problem}{1.5}
\textbf{[14pts]} Some of the following arguments are valid by universal modus ponens or universal modus tollens; others are invalid and exhibit the converse or the inverse error. State which are valid and which are invalid. Justify your answers.
\begin{enumerate}[a)]
\item All healthy people eat an apple a day. Keisah eats an apple a day. So, Keisha is a healthy person.
\item All freshman must take a writing course. Caroline is a freshman. So, Caroline must take a written course. 
\item If a product of two numbers is 0, then at least one of the numbers is 0. I pick a number $x$ such that neither $(2x+1)$ nor $(x-7)$ equals 0. Therefore, the product $(2x+1)(x-7)$ is not 0.
\item All cheaters sit in the back row. Monty sits in the back row. So, Monty is a cheater. 
\item Any sum of two rational numbers is rational. The sum $r+s$ is rational. Therefore, the numbers $r$ and $s$ are both rational. 
\item If a number is even, then twice that number is even. I pick a number and tell you that twice it is even. Therefore, my even is even. 
\item If a computer program is correct, then compilation of the program does not produce errors. Compilation of your amazing program does not produce error messages. Therefore, your program is correct!
\end{enumerate}
\end{problem}

\begin{solution}
% your answer here
\end{solution}
\newpage % delete to remove white space

%%% Problem 6
\begin{problem}{1.6}
\textbf{[7pts]} The computer scientists Richard Conway and David Gries, both very interesting people you should look up, once wrote: 

\textit{\textbf{``The absence of error messages during translation of a computer program is only a necessary and not a sufficient condition for reasonable [program] correctness."}}

Rewrite this statement without using the words \textit{necessary} or \textit{sufficient} to convey your understanding of the ideas.
\end{problem}
\begin{solution}
% your solution here
\end{solution}
\vspace{2in} % delete to remove white space

\begin{problem}{1.7}
\textbf{[15pts]} Let the following be accepted premises:
\begin{enumerate}[i.]
\item $\forall x (P(x) \lor Q(x))$
\item $\forall x (\neg Q(x) \lor S(x))$
\item $\forall x(R(x) \Rightarrow \neg S(x))$
\item $\exists x \neg P(x)$
\end{enumerate}
Use the rules of inference to show that $\exists x \neg R(x)$ follows from these premises, being sure to state each rule used for each step. 
\end{problem}

\begin{solution}
% your solution here
\end{solution}
\newpage % delete to remove white space


%% Problem 8
\begin{problem}{1.8}
\textbf{[10 pts]} Consider the following expression:
$$ \neg(\forall\epsilon\exists\delta((f(x)<\epsilon)\rightarrow(g(x)<\delta))) $$
Use DeMorgan's Law(s) to rewrite this statement without any $\neg$ sign. (Hint: the negation of $<$ is $\geq$.)
\end{problem}
\begin{solution}
$ Write your answer here$
\end{solution}
\newpage

%%% Problem 9
\begin{problem}{1.9}
\textbf{[10pts]} \textbf{Constraint Satisfaction Problems} (CSPs) are mathematical problems where decisions must be made to satisfy a set of constraints or limitations. In class, we talked about Sudoku as a game with a set of constraints on the rows, columns, boxes, and existing numbers. To win, you must satisfy them all and thus solve a constraint satisfaction problem. 

\vspace{1em}

Another example of a CSP is class scheduling. You have a set of teachers, student preferences, and other limitations and must create a class schedule that meets them (or if they're not satisfiable that minimizes how many limitations you violate). You can read more about this in ``Course Scheduling As A Constraint Satisfaction Problem" by Juan Jose Blanco and Lina Khatib (\url{https://bit.ly/2vM8y3I}). 

\vspace{1em}

Describe another example of a constraint satisfaction problem. Discuss how you might write down the constraints using the logical formalism we have learned (provide at least 3 examples with at least one quantified statement). Read about how hard it is to generally solve constraint satisfaction problems and indicate your findings. Your answer should be a couple paragraphs.
\end{problem}
\begin{solution}
% your answer here
\end{solution}
\vspace{4in} % delete to remov4e white space

\begin{problem}{Reflection}
\textbf{[3pts]} Exercise a growth mentality by reflecting on this assignment and your work. Feel free to say whatever you want, but you are required to answer the following. You are graded on whether you complete this, not on what you say.
\begin{itemize}
\item How many hours did you spend on this assignment?
\item What problem was hardest? Why?
\item What problem was easiest? Why?
\end{itemize}
\end{problem}

\end{document}